\documentclass{article}

\usepackage{xcolor}
\usepackage{graphicx}
\graphicspath{ {./figures/} }
\usepackage{amsmath}
\usepackage{amssymb}
\usepackage{amsthm}
\usepackage{float}
\usepackage{placeins}
\usepackage{hyperref}
\usepackage{cleveref}
\usepackage{booktabs}
\usepackage{minted}
\usepackage{csquotes}

\newcommand{\assoc}{\oplus}
\newcommand{\conj}{\wedge}
\newcommand{\disj}{\vee}
\newcommand{\concat}{\ensuremath{\mathbin{+\mkern-10mu+}}}
\newcommand{\compose}{\ensuremath{\mathbin{\circ}}}
\newcommand{\map}{\ensuremath{\mathbin{\ast}}}
\newcommand{\crossop}{\ensuremath{\mathbin{\mathcal{X}}}}
\newcommand{\maxop}{\uparrow}
\newcommand{\minop}{\downarrow}


\begin{document}

\title{DPP Assignment 3}
\author{Cornelius Sevald-Krause \\ \texttt{<lgx292>}}
\date{2022-12-16}
\maketitle

\section*{Task 1}

The code for \verb|flatIf| is not shown as it's a quite large function but can
be found in the \verb|flat-if-then-else.fut| file. Run \verb|make test-if| to
test it and \verb|make bench-if| to run the benchmarks. You can optionally
choose a different backend (defaults to \verb|opencl|) when testing/benchmarking
e.g. with\\ \verb|make BACKEND=c bench-if|.

As an additional test, I selected following (nested) input:
\begin{Verbatim}
shp = [0,  1,   2,     3,       4,         5          ]
xss = [[], [1], [2,2], [3,3,3], [4,4,4,4], [5,5,5,5,5]]
bs  = [F,  T,   F,     F,       T,         T          ]
\end{Verbatim}
As can be seen, \verb|f| is mapped over \verb|[1]|, \verb|[4,4,4,4]| and
\verb|[5,5,5,5,5]| and \verb|g| is mapped over \verb|[]|, \verb|[2,2]| and
\verb|[3,3,3]| which should produce:
\begin{Verbatim}
xss_res = [[], [2], [4,4], [6,6,6], [5,5,5,5], [6,6,6,6,6]]
\end{Verbatim}

For the large dataset, I am creating \num{10000000} (ten million) random inputs
for \verb|xss| and \num{1000} random inputs for \verb|bs|. I'm then creating
the shape array as a uniform array filled with \num{1000} \verb|10000|-values.
(as $\num{1000} \cdot \num{10000} = \num{10000000}$).

The sequential version does much better against the parallel version for small
subarrays so \num{1000} subarrays of size \num{10000} was chosen as a middle
ground. With these parameters, the parallel version gives a speedup of
\textapprox 46.16x. ($\qty{17174}{\micro\second}$ vs.
$\qty{792759}{\micro\second}$ for \verb|opencl| vs. \verb|c| resp.). The
benchmarks were done with Futhark version 0.21.4 on the \verb|gpu04-diku-apl|
machine.

\section*{Task 2}

\textbf{Intuition:} we want to increment the values of \verb|iss| so that they
match the indices of \verb|xss| they are meant to reference. To do this, we'll
probably need the shape of \verb|xss| to know how much we should increment. Each
subarray in \verb|iss| should be incremented the same amount, which suggests we
should use the Inner Indices of \verb|iss|.

This leads to the following steps:
\begin{enumerate}
    \item Do an exclusive prefix sum over the shape array of \verb|xss|.
    \item Create the Inner Indices ($\texttt{II}^1_{\texttt{iss}}$) array of the
        shape of \verb|iss|/\verb|vss|.
    \item Pairwise add the values of \verb|iss| and the array computed in step 1
        referenced through $\texttt{II}^1_{\texttt{iss}}$ i.e. $\texttt{iss'}[i]
        = \texttt{iss}[i] + \texttt{scn}_{\texttt{shp}} \left[
            \texttt{II}^1_{\texttt{iss}}[i] \right]$ where
        $\texttt{scn}_{\texttt{shp}}$ is the array from step 1.
    \item Do a \verb|scatter| over \verb|xss|, \verb|iss'| and \verb|vss| where
        \verb|iss'| is the array from step 3.
\end{enumerate}

% debating if i should keep this section
\ignore{
Instead of computing $\texttt{II}^1_{\texttt{iss}}$ it is possible to simply
construct a flag array from the shape of \verb|iss| with the values of the flags
being the array from step 1 and then doing a segmented scan with the flag array
acting both as the actual flags and array to scan over and then directly doing a
pairwise addition over the \verb|iss| and the scanned flag array, which saves a
level of indirection through $\texttt{II}^1_{\texttt{iss}}$. I chose not to do
this as $\texttt{II}^1_{\texttt{iss}}$ might already be computed from earlier
flattening.
}

\end{document}
